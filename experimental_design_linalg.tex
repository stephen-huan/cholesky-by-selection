% SIAM Article Template

% Information that is shared between the article and the supplement
% (title and author information, macros, packages, etc.) goes into
% ex_shared.tex. If there is no supplement, this file can be included
% directly.

% SIAM Shared Information Template
\documentclass[preprint,onefignum,onetabnum]{siamart220329}

% This is information that is shared between the main document and any
% supplement. If no supplement is required, then this information can
% be included directly in the main document.


% Packages and macros go here
\usepackage{lipsum}
\usepackage{amssymb}
\usepackage{amsfonts}
\usepackage{graphicx}
\usepackage{epstopdf}
\usepackage{algorithmic}
\usepackage{tikz}
\usepackage{pgfplots}
\usetikzlibrary{shapes.arrows, patterns, calc}
\usepackage{tikz-3dplot}
\usepackage{bbm}
\usepackage{bm}
\ifpdf
  \DeclareGraphicsExtensions{.eps,.pdf,.png,.jpg}
\else
  \DeclareGraphicsExtensions{.eps}
\fi
\usepackage{todonotes}

\definecolor{lightblue}{HTML}{a1b4c7}
\definecolor{orange}{HTML}{ea8810}
\definecolor{silver}{HTML}{b0aba8}
\definecolor{rust}{HTML}{b8420f}
\definecolor{seagreen}{HTML}{23553c}

\colorlet{lightsilver}{silver!30!white}
\colorlet{darkorange}{orange!85!black}
\colorlet{darksilver}{silver!85!black}
\colorlet{darklightblue}{lightblue!85!black}
\colorlet{darkrust}{rust!85!black}
\colorlet{darkseagreen}{seagreen!85!black}

\hypersetup{
  colorlinks=true,
  linkcolor=darkrust,
  citecolor=darkseagreen,
  urlcolor=darksilver
}

\pgfplotsset{compat=newest}
\usepgfplotslibrary{fillbetween}

% custom commands:
% \newcommand{\todo}[1]{{\color{red} #1}}
\newcommand{\Ftodo}[1]{\todo[color=lightblue]{Florian: #1}}
\newcommand{\Stodo}[1]{\todo[color=seagreen]{Stephen: #1}}

\usepackage{mathtools}
%Names of standard objects
\newcommand*{\Reals}{\mathbb{R}}
\newcommand*{\Naturals}{\mathbb{N}}

\makeatletter
\renewcommand{\paragraph}{%
  \@startsection{paragraph}{4}%
  {\z@}{1.0ex \@plus .5ex \@minus .2ex}{-.7em}%
  {\normalfont\normalsize\bfseries}%
}
\makeatother

% From https://tex.stackexchange.com/questions/198771/align-in-substack
\makeatletter
\newcommand{\subalign}[1]{%
  \vcenter{%
    \Let@ \restore@math@cr \default@tag
    \baselineskip\fontdimen10 \scriptfont\tw@
    \advance\baselineskip\fontdimen12 \scriptfont\tw@
    \lineskip\thr@@\fontdimen8 \scriptfont\thr@@
    \lineskiplimit\lineskip
    \ialign{\hfil$\m@th\scriptstyle##$&$\m@th\scriptstyle{}##$\hfil\crcr
      #1\crcr
    }%
  }%
}
\makeatother

%Names of Variables:
\newcommand*{\stateP}{\mathbf{x}}
\newcommand*{\stateD}{\hat{\mathbf{x}}}

\newcommand*{\resP}{\mathbf{r}}
\newcommand*{\resD}{\hat{\mathbf{r}}}

\newcommand*{\resMatP}{\mathbf{R}}
\newcommand*{\resMatD}{\hat{\mathbf{R}}}


\newcommand*{\dirP}{\mathbf{p}}
\newcommand*{\dirMatP}{\mathbf{P}}
\newcommand*{\dirD}{\hat{\mathbf{p}}}
\newcommand*{\dirMatD}{\hat{\mathbf{P}}}

\newcommand*{\rhsP}{\mathbf{b}}
\newcommand*{\rhsD}{\hat{\mathbf{b}}}

\newcommand*{\mat}{\mathbf{A}}
\newcommand*{\adj}[1]{#1^{\dagger}}
\newcommand*{\con}[1]{\bar{#1}}

\newcommand*{\alphaP}{\alpha}
\newcommand*{\alphaD}{\hat{\alpha}}

\newcommand*{\alphaMatP}{\bm{\alpha}}
\newcommand*{\alphaMatD}{\hat{\bm{\alpha}}}

\newcommand*{\betaP}{\beta}
\newcommand*{\betaD}{\hat{\beta}}

\newcommand*{\betaMatP}{\bm{\beta}}
\newcommand*{\betaMatD}{\hat{\bm{\beta}}}

\renewcommand{\vec}[1]{\bm{#1}}

\renewcommand{\algorithmicrequire}{\textbf{Input:}}
\renewcommand{\algorithmicensure}{\textbf{Output:}}

% Add a serial/Oxford comma by default.
\newcommand{\creflastconjunction}{, and~}

% Used for creating new theorem and remark environments
\newsiamremark{remark}{Remark}
\newsiamremark{hypothesis}{Hypothesis}
\newsiamremark{example}{Example}
\crefname{hypothesis}{Hypothesis}{Hypotheses}
\newsiamthm{claim}{Claim}

% Sets running headers as well as PDF title and authors
\headers{Factorization by greedy conditional selection}{
S. Huan, J. Guinness, M. Katzfuss, H. Owhadi, F. Sch{\"a}fer}

% Title. If the supplement option is on, then "Supplementary Material"
% is automatically inserted before the title.
\title{Sparse Cholesky factorization by greedy conditional selection}

% Authors: full names plus addresses.
\author{
  Stephen\ Huan\thanks{Georgia Institute of Technology} \and
  Joe\ Guinness\thanks{Joe} \and
  Matthias\ Katzfuss\thanks{Matthias} \and
  Houman\ Owhadi\thanks{Houman} \and
  Florian\ Sch{\"a}fer\thanks{Georgia Institute of Technology,
  S1317 CODA, 756 W Peachtree St Atlanta, GA 30332, \newline
  \email{florian.schaefer@cc.gatech.edu},\newline Corresponding Author}
}


\usepackage{amsopn}
\DeclareMathOperator{\diag}{diag}
\let\trace\relax
\DeclareMathOperator{\trace}{trace}
\DeclareMathOperator{\logdet}{logdet}
\DeclareMathOperator{\chol}{chol}

\DeclareMathOperator*{\argmin}{argmin}

\DeclareMathOperator{\E}{E}
\DeclareMathOperator{\Var}{Var}
\DeclareMathOperator{\Cov}{Cov}
\DeclareMathOperator{\Corr}{Corr}
\DeclareMathOperator{\I}{MI}
\DeclareMathOperator{\entropy}{H}

%%% Local Variables:
%%% mode:latex
%%% TeX-master: "ex_article"
%%% End:


% Optional PDF information
\ifpdf 
\hypersetup{
  pdftitle={m-calculus},
  pdfauthor={S Huan, F. Sch{\"a}fer}
}
\fi

% The next statement enables references to information in the
% supplement. See the xr-hyperref package for details.

\externaldocument{experimental_linalg_supplement}

% FundRef data to be entered by SIAM
%<funding-group specific-use="FundRef">
%<award-group>
%<funding-source>
%<named-content content-type="funder-name"> 
%</named-content> 
%<named-content content-type="funder-identifier"> 
%</named-content>
%</funding-source>
%<award-id> </award-id>
%</award-group>
%</funding-group>

\begin{document}

\maketitle

% REQUIRED
\begin{abstract}
  experimental design for linear algebra 
\end{abstract}

% REQUIRED
\begin{keywords}
   \todo{to do}
\end{keywords}

% REQUIRED
\begin{AMS}
  % 68Q25, 68R10, 68U05
\end{AMS}

\section{Introduction}
\cite{schafer2017compression} test citation

\section{Greedy selection for directed inference}

A \emph{Gaussian process} is a prior distribution over functions, such that
for any finite set of points, the corresponding function over the points
is distributed according to a multivariate Gaussian. In order to generate
such a distribution over an uncountable number of points consistently, a
Gaussian process is specified by a \emph{mean function} \( \mu(\bm{x}) \) and
\emph{covariance function} or \emph{kernel function} \( k(\bm{x}, \bm{y})
\). For any finite set of points \( X = \{ x_1, \dots x_n \} \), \( f(X)
\sim \mathcal{N}(\bm{\mu}, \Theta) \), where \( \bm{\mu}_i = \mu(X_i) \) and
\( \Theta_{ij} = k(X_i, X_j) \). We are focused on performing regression
with Gaussian processes. This means we have a set of training data points \(
X_\text{Tr} \) and their corresponding function values \( \bm{y}_\text{Tr}
\) and want to make predictions at points \( X_\text{Pr} \). To do so,
we can simply condition the desired prediction \( \bm{y}_\text{Pr} \) on
the observed outputs and compute the conditional expectation. We can also
find the conditional variance, which will quantify the uncertainty of our
prediction. If we block our covariance matrix
\( \Theta = 
  \begin{pmatrix} 
    \Theta_{\text{Tr}, \text{Tr}} & \Theta_{\text{Tr}, \text{Pr}} \\
    \Theta_{\text{Pr}, \text{Tr}} & \Theta_{\text{Pr}, \text{Pr}}
  \end{pmatrix}
\) 
where \( \Theta_{\text{Tr}, \text{Tr}}, \Theta_{\text{Pr}, \text{Pr}},
\Theta_{\text{Tr}, \text{Pr}}, \Theta_{\text{Pr}, \text{Tr}} \) are the
covariance matrices of the training data, testing data, and training and test
data respectively, then the conditional expectation and covariance are:
\begin{align}
  \label{eq:cond_mean}
  \E[\bm{y}_\text{Pr} \mid \bm{y}_\text{Tr}] &= \bm{\mu}_\text{Pr}
    + \Theta_{\text{Pr}, \text{Tr}} \Theta_{\text{Tr}, \text{Tr}}^{-1} 
      (\bm{y}_\text{Tr} - \bm{\mu}_\text{Tr}) \\
  \label{eq:cond_cov}
  \Cov[\bm{y}_\text{Pr} \mid \bm{y}_\text{Tr}] &= \Theta_{\text{Pr}, \text{Pr}}
    - \Theta_{\text{Pr}, \text{Tr}} \Theta_{\text{Tr}, \text{Tr}}^{-1} 
      \Theta_{\text{Tr}, \text{Pr}}
\end{align}

Note that calculating the posterior mean and variance requires inverting the
training covariance matrix \( \Theta_{\text{Tr}, \text{Tr}} \), which costs
\( \mathcal{O}(N^3) \), where \( N \) is the number of training points. This
scaling is prohibitive for large datasets, so many \emph{sparse} Gaussian
process regression techniques have been proposed. These methods often focus
on selecting a subset of the training data that is most informative about the
prediction points. If \( s \) points are selected out of the \( N \), then
the inversion will cost \( \mathcal{O}(s^3) \), which could be substantially
cheaper if \( s \) is significantly smaller than \( N \). The question is then
how to select as few points as possible while maintaining predictive accuracy.

\todo{cite sparse Gaussian regression papers}

\subsection{Problem: optimal selection}

One natural criterion is to maximize the \emph{mutual information}
between the selected points and the target point for prediction. Such
information-theoretic objectives have seen success in the spacial statistics
community \cite{krause2008sensor}, who use such criteria to determine the best
locations to place sensors in a Gaussian process regression context.
The mutual information, or \emph{information gain} is defined as
\begin{align}
  \label{eq:info}  
  \I[\bm{y}_\text{Pr};\bm{y}_\text{Tr}] &=  \entropy[\bm{y}_\text{Pr}] - 
    \entropy[\bm{y}_\text{Pr} \mid \bm{y}_\text{Tr}]
\end{align}

We can use the fact that the entropy of a multivariate Gaussian is
monotonically increasing with the log determinant of its covariance
matrix to efficiently compute these entropies. Because the entropy of \(
\bm{y}_\text{Pr} \) is constant, maximizing the mutual information is
equivalent to minimizing the conditional entropy. From \cref{eq:cond_cov}
we see that minimizing the conditional entropy is equivalent to minimizing
the log determinant of the posterior covariance matrix. Note that for a
single predictive point, this is monotonic with its variance. So another
justification is that we are reducing the \emph{conditional variance} of
the desired point as much as possible. In particular, because our estimator
is the conditional expectation \cref{eq:cond_mean}, it is unbiased because
\( \E[\E[\bm{y}_\text{Pr} \mid \bm{y}_\text{Tr}]] = \E[\bm{y}_\text{Pr}]
\). Because it is unbiased, its expected mean squared error is simply the
conditional variance since \( \E[(\bm{y}_\text{Pr} - \E[\bm{y}_\text{Pr}
\mid \bm{y}_\text{Tr}])^2 \mid \bm{y}_\text{Tr}] = \Var[\bm{y}_\text{Pr}
\mid \bm{y}_\text{Tr}] \) where the expectation is taken under conditioning
because of the assumption that \( \bm{y}_\text{Pr} \) is distributed
according to the Gaussian process. So maximizing the mutual information is
equivalent to minimizing the conditional variance which is in turn equivalent
to minimizing the expected mean squared error of the prediction.

\subsection{A greedy approach}

We now consider how to efficiently maximize the conditional variance objective
using a greedy approach. At each iteration, we pick the training point which
most reduces the conditional variance of the prediction point. Let \( I =
\{ i_1, i_2 \dots i_j \} \) be the set of indexes of training points selected
already. For a candidate index \( k \), \todo{elaborate on this argument}, because iterative conditioning of
the Gaussian process is Schur complementation and Cholesky factorization is
recursive Schur complementation, we see that the amount selecting \( k \)
will decrease the variance of \( \bm{y}_\text{Pr} \) is 
\begin{align}
  \label{eq:gp_greedy}
  \frac{\Cov[\bm{y}_\text{Pr}, \bm{y}_{\text{Tr},k}]^2}
       {\Var[\bm{y}_{\text{Tr}, k}, \bm{y}_{\text{Tr}, k}]} &=  
  \frac{\Theta_{nk \mid I}^2}{\Theta_{kk \mid I}}              
\end{align}
We can efficiently keep track of the conditional variance of each training
point as well as the conditional covariance of each training point with the
prediction point by maintaining a partial Cholesky factorization from the
indices \( I \) already selected. 
\todo{finish describing algorithm(s), proofs, psuedocode}

\subsection{Near optimality by submodularity}

\section{Greedy selection for \emph{global} approximation by KL-minimization}

We have a symmetric, positive (semi-)definite kernel matrix \( \Theta \) and
wish to compute the \emph{Cholesky factorization} of \( \Theta \) into a lower
triangular factor \( L \) such that \( \Theta = L L^{\top} \). \todo{justify
importance/applications of Cholesky factorization}. This can be done in \(
\mathcal{O}(N^3) \) with standard algorithms, which is often prohibitive.
Instead, we want to do so in a \emph{sparse} manner. As such, we can only get
an approximate \( L \), \( \hat{L} \) belonging to a pre-specified sparsity
pattern \( S \) (a set of (row, column) indices that are allowed to be
nonzero). In order to measure the performance of the estimator, we imagine both
kernel matrices as the covariance matrices of multivariate Gaussians with mean
\( \bm{0} \). In order to compare the resulting two distributions, we use the
\emph{KL divergence} according to \cite{schafer2020sparse}, or the difference
in information it takes to encode a sample from \( \Theta \) with a coding
scheme from \( (\hat{L} \hat{L}^{\top})^{-1} \), compared to encoding with the
optimal coding scheme using \( \Theta \). \todo{use different interpretation
of KL divergence --- explicit connection to Gaussian process regression}
\begin{align}
  \label{eq:L_obj} 
  L \coloneqq \argmin_{\hat{L} \in S} \, \mathbb{D}_{\text{KL}}
  \left (
    \mathcal{N}(\vec{0}, \Theta) \, \Big \| \,
    \mathcal{N}(\vec{0}, (\hat{L} \hat{L}^{\top})^{-1})
  \right ) 
\end{align}

Note that here we are computing the Cholesky factorization
of \( \Theta^{-1} \). Surprisingly enough, it is possible to
exactly compute \( L \). First, we re-write the KL divergence:
\begin{align}
  \label{eq:KL}  
  2 \mathbb{D}_{\text{KL}}
  \left (
    \mathcal{N}(\vec{0}, \Theta_1) \, \Big \| \,
    \mathcal{N}(\vec{0}, \Theta_2)
  \right ) 
  = \trace(\Theta_2^{-1} \Theta_1) + \logdet(\Theta_2) - \logdet(\Theta_1) - N
\end{align}

\begin{theorem}
  \cite{schafer2020sparse}. 
  The non-zero entries of the \( i \)th
  column of \( L \) in \cref{eq:L_obj} are:
  \begin{align}
    \label{eq:L_col} 
    L_{s_i, i} = \frac{\Theta_{s_i, s_i}^{-1} \vec{e}_1}
    {\sqrt{\vec{e}_1^{\top} \Theta_{s_i, s_i}^{-1} \vec{e}_1}}
  \end{align}
\end{theorem}

Plugging \( L \) \cref{eq:L_col} back into
the KL divergence \cref{eq:KL}, we obtain:
\begin{align}
  \mathbb{D}_{\text{KL}}
  \left (
    \mathcal{N}(\vec{0}, \Theta) \, \Big \| \,
    \mathcal{N}(\vec{0}, (L L^{\top})^{-1})
  \right ) 
  &= -\logdet(L L^{\top}) - \logdet(\Theta) \nonumber
  \shortintertext{Because  \( L \) is lower triangular, its
    determinant is just the product of its diagonal entries:} \nonumber
  &= -2 \sum_{i = 1}^N \log(L_{ii}) - \logdet(\Theta) 
  \shortintertext{Plugging \cref{eq:L_col} for its diagonal entry,}
  \label{eq:obj}
  &= -\sum_{i = 1}^N \log(\bm{e}_1^{\top} \Theta_{s_i, s_i}^{-1} \bm{e}_1) 
    - \logdet(\Theta) 
\end{align}

In order to maximize \cref{eq:obj}, we can maximize over each column
independently, since each term in the sum only depends on a single column.
We want to maximize \( \bm{e}_1^{\top} \Theta_{s_i, s_i}^{-1} \bm{e}_1
\), the term corresponding to the diagonal entry in the inverse of the
submatrix of \( \Theta \) corresponding to the entries we've taken.

Given a sparsity pattern, we can construct \( L \). The question is now how to
compute a good sparsity pattern such that the resulting \( L \) has as small
KL divergence as possible. First, we can consider a single column since the KL
divergence is independent between columns. Throughout this discussion we'll
assume we're selecting at most \( s \) nonzero entries from each column. In
order to select nonzero rows for a single column, we can do this greedily, by
picking the row that locally decreases the KL divergence the most. Once we've
selected it, we take the next entry according to the same criteria, and so on.

However, if we were to do this naively, we would iterate over the \( N \)
possible candidate indices \( k \). For each \( k \), we would take the
new sparsity pattern \( s_i' = s_i \cup \{k\} \), and compute the inverse
of the submatrix \( \Theta \) indexed at this new sparsity pattern, \(
\Theta_{s_i', s_i'}^{-1} \). Finally, we would pick the \( k \) with the
largest top left entry and add it to our sparsity pattern. This would
cost \( s^3 \) per candidate to invert the resulting matrix, over \( N
\) candidates and \( s \) rounds, with a cost of \( N s^4 \) per column
and \( N^2 s^4 \) over all the columns. Luckily, we can do much better
by taking advantage of Schur complements.

We can cleverly block our matrix in a way to take advantage of the redundancy
in computing \( \Theta_{s_i', s_i'}^{-1} \). Note that we can assume we have \(
\Theta_{s_i, s_i}^{-1} \), the inverse of the entries we've selected already.
Then when we consider \( k \) as a candidate, we're just adding a single row
and column to this new matrix. If we organize everything, our \( \Theta_{s_i',
s_i'} \) will be:
\begin{align}
  \label{eq:block}
  P \Theta_{s_i', s_i'} P^{\top} &= 
  \left ( 
  \begin{array}{cc|c}
    \cline{1-1}
    \multicolumn{1}{|c|}{\Theta_{11}} & \Theta_{12} & \Theta_{13} \\
    \cline{1-1}
    \Theta_{21} & \Theta_{22} & \Theta_{23} \\ \hline
    \Theta_{31} & \Theta_{32} & \Theta_{33}
  \end{array} 
  \right )
\end{align}

Here \( \Theta_{33} \) is the diagonal entry \( \Theta_{ii} \), henceforth
known as the \emph{special entry}, \( \Theta_{11} \) are the entries we've
taken already (excluding \( i \)), and \( \Theta_{22} \) is \( \Theta_{kk}
\), the candidate entry. The off diagonal terms all correspond to the proper
indexing of \( \Theta_{s_i', s_i'} \), i.e. \( \Theta_{21} \) is \( \Theta \)
at row \( k \) indexed along the sparsity pattern \( s_i \), \( \Theta_{31} \)
is \( \Theta \) at row \( i \) indexed along \( s_i \), and \( \Theta_{32} \)
is the scalar \( \Theta_{ik} \). The terms above the diagonal are symmetric.

Thus, computing \( (\Theta_{s_i', s_i'}^{-1})_{11} \) is equivalent to
computing \( ((P \Theta_{s_i', s_i'} P^{\top})^{-1})_{33} \) since inverting
a permuted matrix just permutes its inverse. We can efficiently compute the
bottom right entry of the block matrix with Schur complementation:
\begin{align}
  \nonumber
  (S/S_{22}) &= (S_{33} - S_{32} S_{22}^{-1} S_{23})^{-1} \\
  (\Theta_{s_i', s_i'}^{-1})_{11} &= 
  \left (
           \Theta_{33} - \Theta_{31} \Theta_{11}^{-1} \Theta_{13} - 
    \frac{(\Theta_{32} - \Theta_{31} \Theta_{11}^{-1} \Theta_{12})^2} 
         { \Theta_{22} - \Theta_{21} \Theta_{11}^{-1} \Theta_{12}}
  \right )^{-1}
  \nonumber
  \shortintertext{Recall that we want to
    maximize this term. So we can minimize}
  \nonumber
 & \equiv \min_k \,  
 \Theta_{33} - \Theta_{31} \Theta_{11}^{-1} \Theta_{13} - 
       \frac{( \Theta_{32} - \Theta_{31} \Theta_{11}^{-1} \Theta_{12})^2} 
             { \Theta_{22} - \Theta_{21} \Theta_{11}^{-1} \Theta_{12}}
  \nonumber
  \shortintertext{Since \( \Theta_{33} - \Theta_{31}
    \Theta_{11}^{-1} \Theta_{13} \) is constant over our candidates,}
 & \equiv \max_k
 \label{eq:chol_greedy}
 \frac{( \Theta_{32} - \Theta_{31} \Theta_{11}^{-1} \Theta_{12})^2} 
       { \Theta_{22} - \Theta_{21} \Theta_{11}^{-1} \Theta_{12}}
\end{align}
This is precisely the objective in \cref{eq:gp_greedy} as the numerator is the
squared covariance between the candidate and the special entry, conditional
on the entries already selected, while the denominator is the conditional
variance of the candidate. Hence the sparse Cholesky factorization motivated
by KL divergence can be viewed as the sparse Gaussian process regression over
each column, where entries are selected to maximize mutual information with
the entry on the diagonal of the current column.
\todo{elaborate on the connection and recycle algorithm}

\subsection{Algorithms for Efficient Schur Complementation}

There are two primary ways we can efficiently compute these quadratic forms.

\subsubsection{Explicit Inverse}

Maintain \( \Theta_{11}^{-1} \) explicitly. When an index \( k \) is added
to the sparsity set, efficiently update \( \Theta_{11}^{-1} \) with Schur
complementation in \( \mathcal{O}(s^2) \). Finally, compute the quadratic
forms by storing the previous values for each candidate and updating based
on the fact that adding a new row and column only adds \( \mathcal{O}(s)
\) terms to compute. For each of the \( s \) rounds, it takes \( s^2 \) to
update the inverse and for each of the \( N \) candidates, it takes \( s \)
to compute their new quadratic form, costing \( N s^2 + s^3 \) per column
and \( \mathcal{O}(N^2 s^2 + N s^3 ) \) overall.

\subsubsection{Cholesky Factorization}

Maintain a Cholesky factorization of \( \Theta_{11} \), \( \Theta_{11} = L
L^{\top} \). When an index is added, update the Cholesky factorization with
left-looking in \( \mathcal{O}(N s) \). Each quadratic form can be updated
in \( \mathcal{O}(1) \) given the Cholesky factorization, yielding \( N s^2
\) per column and \( \mathcal{O}(N^2 s^2) \) overall.

While both approaches have similar time complexity, the explicit inverse
algorithm uses \( \mathcal{O}(s^2 + N) \) space to store the inverse and
quadratic forms while the Cholesky factorization uses \( \mathcal{O}(N s)
\) to store the Cholesky factors (\( N > s \)). Also the explicit inverse
algorithm computes \( \Theta_{11}^{-1} \), which can be directly used to
generate the columns of \( L \) according to \cref{eq:L_col} without extra
computational cost.

\subsection{Review of KL approximation}

\subsection{Supernodes and blocked selection}

\section{Numerical experiments} 

\section*{Acknowledgments}
%\todo{add more funding information}
This research was supported in part through research cyberinfrastructure resources and services provided by the Partnership for an Advanced Computing Environment (PACE) at the Georgia Institute of Technology, Atlanta, Georgia, USA.

\bibliographystyle{siamplain}
\bibliography{references}

\appendix
\todo{add proofs, if any, in appendix}

\end{document}
